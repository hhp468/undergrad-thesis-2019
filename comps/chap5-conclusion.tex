The introduction section showed that there are several obstacles that 
prevent us from getting a research direction towards the next generation 
of antifreeze. In this study, I suggested that such problems can be overcome 
by creating a function that connects the antifreeze structural features to 
its thermophysical properties such as thermal conductivity and viscosity. 
The function is created by implementing a supervised machine learning method 
called xgboost while the data used in training xgboost models is obtained 
from Green-Kubo relation based molecular dynamics simulations of 8 types of 
alcohol: methanol, ethanol, ethylene glycol, glycerol, 1-propanol, 2-propanol, 
1,3-propanediol and propylene glycol. Such a function helps us to gain clues 
for constructing the next generation of alcohol coolants while it also shows 
us a research direction to optimize this method in the future.
\section*{Summary of Findings}
The training data is decided by 5 different ways: randomly divided, 
excluding methanol, excluding 2-propanol, excluding glycerol and excluding 
propylene glycol. The results showed that the models can predict values that 
have similar patterns to training data, while they had difficulty predicting 
properties that were very different to the training data. Thus, to overcome 
this problem in the future, there are 2 different approaches: diversification 
of the training data and conducting of research on special alcohol types such 
as methanol.\\
The machine learning approach can also statistically show that besides factors 
such as the concentration and number of hydroxyl groups, the hydroxyl group at 
the 3rd carbon position in the main carbon chain can significantly affect the 
thermal conductivity values. Similarly, the hydroxyl group at the 2nd carbon 
position in the main carbon chain has a significant influence on the viscosity 
values.\\
For the given alcohol sample excluding methanol, the optimization predicted 
that the next alcohol antifreeze candidate would possess 3 to 4 carbons, 3 
oxygens, 13 or 17 single bonds, a main carbon chain of length 3 and have 
hydroxyl groups in every position. However, the concentration showed that 
the mixture is almost water, which makes sense since we are only considering 
the thermal conductivity and viscosity. Furthermore, one of molecular 
structures that can be constructed from the predicted structural features is 
glycerol, the type of alcohol that has the highest thermal conductivity and 
fairly low viscosity across mass concentration of alcohol-water mixtures 
according to simulated data. Thus, the results show that the optimization 
algorithm approaches the problem in an appropriate manner. With the addition 
of freezing point in the future, the methodology will be able to produce 
desired outputs. 
Furthermore, the prediction is likely to be further improved with a more 
diverse sample data. An interpreting method is also needed to translate the 
obtained structural feature values to a complete final structure.
