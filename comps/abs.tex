\thispagestyle{plain}
\begin{center}
    \renewcommand{\baselinestretch}{1.5}
    \Large
    \textbf{TOWARDS PREDICTION OF OPTIMAL ALCOHOL ANTIFREEZE STRUCTURAL 
    FEATURES USING THE EXTREME GRADIENT BOOSTING METHOD WITH OPTIMIZATION}
 
    %\vspace{0.4cm}
    %\large
    %Thesis Subtitle
 
    \vspace{0.4cm}
    \large
    \textbf{Phong Ho}
 
    %\vspace{0.9cm}
    %\textbf{Abstract}
\end{center}

In this research, we used a statistical approach to find a research direction 
as well as to predict structural features of the next generation of alcohol 
coolant, studying from 8 types of alcohol: methanol, ethanol, ethylene glycol, 
1-propanol, 2-propanol, glycerol, 1,3-propanol and propylene glycol. 
The statistical approach is done by implementing a supervised machine 
learning technique named extreme gradient boosting while the experimental 
data for the statistical technique is obtained from the Green-Kubo relation 
based molecular dynamics simulations.

The results showed that the machine learning models can predict values from 
what they have learned. However, the models could not perform well for the 
data that has a different pattern than the training data. The results also 
suggest that to solve this problem, these 2 solutions should be taken in 
action: diversify the training data and conduct research on special alcohol 
types.\\
The machine learning approach can also show the influence of structural 
features on thermal conductivity and viscosity, which from that further 
research can be conducted to examine the underlying mechanism of those 
features.\\
The optimization part successfully predicted the structural features of 
the next potential candidate from the given alcohol sample. However, a 
future improvement of variable set and an interpreting method need to be 
developed in order to construct a complete molecular structure.