\section{Research Background}
Choice of a cooling system is inevitable in many applications since the 
involvement of heat transfer is important in many industries such as 
automobiles, architecture and electronics. The cooling system plays 
an important role in managing the system temperature, preventing the 
operating system from being overheated, which could lead to severe 
consequences if it is not taken seriously. Compared to the development 
of cooling technologies, the development of the cooling medium itself is 
often less considered. The choosing of an appropriate coolant would improve 
the performance as well as reduce cost for building the system.

Among liquid coolants, water is considered to be one of the best 
candidates for coolants with its high heat capacity and low viscosity 
\cite{mohapatra_advances_2005}. However, the temperature range of 
water in the form of liquid is only from $0^\circ$C to $100^\circ$C, 
which is a relatively small range. If the system 
temperature gets out of the range, the expansion of ice or the evaporation 
of vapor could cause unexpected behaviors, which can lead to the break down 
of the entire system. 

One of the ideas to overcome this problem is to create an aqueous-based 
solution with an alcohol to expand the operational temperature range while 
keeping the good performance in heat transfer and low viscosity. Many alcohol 
types have been introduced for this approach, such as propylene glycol, 
ethylene glycol, methanol or ethanol. However, their low thermal conductivity 
as well as high viscosity become a burden for the cooling medium. Thus, the 
finding of new alcohol in cooling is necessary. However, while there are an 
enormous number of possibilities, the cost of conducting an experiment is not 
neglectable, not to mention that we do not have a clear direction to research 
towards the goal.

Such problems can be overcome if we can create functions that link 
molecular structure features to the values of desired properties such as 
thermal conductivity and viscosity. With such a function, we can adjust any 
structural feature to obtain corresponding property values as well as study 
further on existing database, we can be correctly oriented to reach the 
optimal coolant that satisfy all the desired property constraints.

Such functions can be obtained with the help of computational power such 
as molecular dynamics simulations and supervised machine learning. 
In this work, we develop a methodology that help us get a research 
direction as well as predict the molecular structural features of 
next-generation alcohol coolants for future antifreeze development 
using molecular simulation code LAMMPS for computer simulations combined 
with supervised machine learning using the eXtreme Gradient Boosting technique.
\section{Why use computational methods?}
To understand why computational power is such a perfect vehicle to deliver 
the answer for this problem, we will introduce the two computational 
techniques and how can they bring great benefits for this type of problem.
\subsection{Supervised machine learning method}
Supervised machine learning is a class of advanced computing and statistical 
techniques that maps a function of given sets of input-output. In other words, 
if the input is information of structural features and the output is desired 
properties, we can obtain the function that we desired as mentioned earlier. 
This is a highlight of implementing the computational method since conducting 
the function by analytical method would take a great amount of time and effort, 
not to mention the difficulty. However, using supervised machine learning 
requires a large amount of data in order to obtain accurate and acceptable 
results. Thus, the perfect combination for machine learning is molecular 
dynamics simulations.
\subsection{Molecular Dynamics simulation}
Molecular dynamics simulation is a computational system that numericalizes 
features and behaviors of atoms and molecules. We will discuss the details 
of molecular dynamics in the next section while here I will give some details 
of notable advantages of molecular dynamics in relation to this problem. 
Compared to manual experimental methods, molecular dynamics simulations 
brings much greater benefits.\\
Molecular dynamics simulations require information of the environment, 
such as system volume, temperature, pressure and numerical information 
of investigated coolants, and appropriate computational power to conduct 
an experiment and calculate thermophysical properties. In that sense, 
it is much safer and precise in terms of environment control as well as 
simpler in terms of experimental setup, which saves a lot of resources.\\
By being able to control the process to every single detail, we also can have 
a more insight of atomic behaviors inside the system, in spite of the 
difficulty of observing it if we conduct the experiment traditionally.\\
Standing out from the above benefit, simulations can significantly reduce the 
time to conduct experiments. The entire process is automated with the human 
factor as an observer which allows the experiments to run continuously, 
avoiding the gap that exists in traditional experiments due to human body 
limitation, yet, without compromising the accuracy. To elevate this ability, 
simulations also can run in parallel, which means we can conduct several 
experiments at the same time; the number depending only on the available 
computational power; and the accuracy is still guaranteed.\\
Thus, we can escalate the number of experiments and get much greater amount 
of data with much shorter time to serve the machine learning purpose.
\newpage\section{Thesis Outline}
The thesis is presented in the following manner:

\textbf{Chapter 1 - Introduction} 
\\Providing the background of the research, the idea 
to tackle the problem that we want to answer as well as the reason why 
we have that idea, an introduction about why computational methods should 
be applied to solve this problem.

\textbf{Chapter 2 - Theory} 
\\Providing information of background knowledge that we need 
to know in order to understand the approach that we are about to propose. 
The background knowledge includes the Green-Kubo formulas of thermal 
conductivity and viscosity, molecular dynamics simulations, and all the 
relevant machine learning related techniques.

\textbf{Chapter 3 - Method} 
\\Providing information of how to operate the approach in 
detail, including cultivating data method and data processing method.

\textbf{Chapter 4 - Results and Discussion} 
\\Providing the results of models 
in different situations, the results from optimization process and discussion 
about the results.

\textbf{Chapter 5 - Conclusion} 
\\Summary the entire work.