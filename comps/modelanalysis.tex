As mentioned in the method section, the training data will be defined in 5 
different ways: randomly divided, excluding all propylene glycol data, 
excluding all glycerol data, excluding all methanol data and excluding all 
2-propanol data. The result for each data division approach is presented as 
follows:
\begin{table}[ht]
    \centering
    \caption{Performance Score on the Testing Data Set}
    \begin{tabular}{|c|c|c|}
        \hline
        \hline
        Training Data & TC $R^2$ Score & Visc $R^2$ Score\\
        \hline
        Randomly divided & 0.96  & 0.70\\
        Propylene Glycol data excluded & 0.86  & 0.74\\
        Glycerol data excluded & 0.36  & 0.76\\
        Methanol data excluded & 0.92 & Failed\\
        2-propanol data excluded & 0.90  & 0.72\\
        \hline
    \end{tabular}
    \label{table:1}
\end{table}

The results indicate that for any alcohol that has structural similarities 
with data in the training set, has a high performance score when used on 
the testing data set. However, the alcohols that are significantly different 
from the training sets have a quite low performance scores.

For thermal conductivity models, except for glycerol-excluded approach model, 
all the models possess a decent performance score. The explanation for this is 
that the testing sets in such cases are similar to the training sets. While 
propylene glycol - 1,3-propanediol and 1-propanol - 2-propanol are pairs of 
structural isomers, methanol is also quite close to ethanol. However, even 
though the molecular structure of glycerol is not so distinguished to the 
others, it is the only type of alcohol that had the gradient of the thermal 
conductivity that trends significantly different from the others; not to 
mention that glycerol is the only type of alcohol that possesses 3 hydroxyl 
groups within the molecule. This helps to explain why the glycerol-excluded 
model performance score is quite low.

A similar explanation can be applied to explain the phenomenon in the 
viscosity models. In methanol-excluded model, the model failed to predict 
the viscosity values of methanol. Recalling the simulation data in the 
previous sub-section, methanol was the only type of alcohol that had a 
non-increasing value trend. It came from the fact that within the sample, 
methanol is the only type of alcohol that is less viscous than water. In 
other words, methanol is very different from the other alcohols when looking 
from a viscosity point of view. Thus, the model could not predict the 
viscosity values of methanol since it did not learn any methanol similar 
relationship during the training process. The other models have moderate 
performance scores but are not satisfactory. Since these models included 
methanol in the training process, it is likely that the bizarre viscosity 
of methanol influenced the prediction ability and resulted in non-satisfactory 
performance scores.

For future improvement of this method, the sample alcohol types should be 
diversified to obtain more generalized prediction models. Since low viscosity 
of methanol is a good point, we should include more alcohol types that are 
less viscous than water. However, there might be a chance that the selected 
variables are not sufficient to cover abnormal cases. Thus, independent 
research on how molecular structures influence the thermophysical properties 
of unusual alcohol types like methanol should be conducted to improve the 
variable sets. At the same time, we also need to diversify the sample in terms 
of structural features. Obviously, the sample alcohol types only have single 
bonds and straight carbon chain. Thus, the addition of double bonds, triple 
bonds, circle structures and tree-like carbon chain could help generalize the 
solutions.

While including methanol in the construction of models can give us a better 
direction to prepare data and investigate on a certain direction, I will 
exclude methanol out of the construction of the model in order to investigate 
the capability of this methodology for prediction when performance scores are 
sufficiently high. The expected performance score should be significantly 
higher than the previous version. Thus, the finalized models performance 
scores become
\begin{table}[ht]
    \centering
    \caption{Finalized Models Performance Scores}
    \begin{tabular}{|c|c|}
        \hline
        \hline
        Thermal Conductivity $R^2$ Score & Viscosity $R^2$ Score\\
        \hline
        0.96  & 0.91\\
        \hline
    \end{tabular}
    \label{table:2}
\end{table}

The models also help us to investigate how variables influence the 
thermophysical properties using the F-score method \cite{sasaki_truth_2007}. 
For the final models, the variable influences (feature importance) 
are as follows:

\begin{figure}[ht]
    \includegraphics[width=1\textwidth]{fitc.png}
    \centering
    \captionsetup{justification=centering}
    \caption{Feature Importance of the Thermal Conductivity Model}
\end{figure}
\begin{figure}[ht]
    \includegraphics[width=1\textwidth]{fiv.png}
    \centering
    \captionsetup{justification=centering}
    \caption{Feature Importance of the Viscosity Model}
\end{figure}
\newpage

As expected, the mass concentration is one of the main factors that decides 
the thermal properties of the mixtures by having the number of water molecules 
and alcohol molecules as the highest influential variables. However, due to 
some noises and possibly the lack of better variables in the viscosity model, 
the score of the number of alcohol molecules is slightly underestimated or the 
score of the number of water molecules is slightly overestimated. Nonetheless 
the importance of these 2 variables are still clearly shown.\\
For thermal conductivity, another research \cite{manjunatha_investigation_2017} also showed that 
the number of hydroxyl groups affect the heat transfer ability of the alcohol. 
Interestingly, by using machine learning approach, we can also know that the 
hydroxyl group at the 3rd carbon position in the main carbon chain and the 
number of single bonds also have a significant influence on the thermal 
conductivity.\\
For viscosity, it is the hydroxyl group at the 2nd carbon position in the 
main carbon chain that significantly affect the viscosity values.
