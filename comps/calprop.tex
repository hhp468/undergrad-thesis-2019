In this work, the calculations of properties are carried out by LAMMPS \cite{plimpton_fast_1995}.
Since the investigated objects are water-alcohol mixtures, the fundamental inputs include\\
- Number of alcohol molecules\\
- Number of water molecules\\
- Declaring implemented alcohol molecular model\\
- System temperature T\\
- System volume V\\
- System pressure P\\
TIP4P/Ice \cite{abascal_potential_2005} will be used as the water model throughout all 
mixtures while all alcohol models are based on OPLS-AA \cite{jorgensen_development_1996}, 
which is derived from the Optimized Potentials for Liquid Simulations models 
\cite{jorgensen_opls_1988}.The system pressure is set at 1 atm while the volume of the 
system is set at 40x40x40 $\textup{\AA}^3$ throughout all simulations.\\
Since we are investigating throughout the mass concentration of mixtures, 
we construct a small converter that approximate the number of alcohol 
molecules and water molecules from input mass concentrations. Assuming 
that the sum of volume of water and volume of alcohol equals the system 
volume, we obtain the following:
\begin{equation}
    m_{w}=\rho_{w} V_{w}
\end{equation}
\begin{equation}
    m_{w}=\rho_{w}\left(V-V_{a l c}\right)
\end{equation}
\begin{equation}
    m_{w}=\rho_{w}\left(V-m_{a l c} / \rho_{a l c}\right) \label{eq:1}
\end{equation}
Where $m_w$ is mass of water, $\rho_w$ is density of water, $V_w$ is volume 
of water, $m_{alc}$ is mass of alcohol, $\rho_{alc}$ is density of alcohol and
$V_{alc}$ is volume of alcohol.\\
But then the mass concentration is:
\begin{equation}
    c_{\%}=\frac{m_{alc}}{m_{a k}+m_{alc}}
\end{equation}
Thus
\begin{equation}
    m_{alc}=\frac{c_{\%}}{1-c_{\%}}m_w  \label{eq:2}
\end{equation}
Substitute \ref{eq:2} into \ref{eq:1} we obtain
\begin{equation}
    m_{w}=\rho_{w}\left(V-\frac{c_{\%}}{1-c_{\%}} m_{w} / \rho_{a l c}\right)
\end{equation}
Collect all the terms that contain $m_w$ into one side then eliminate the coefficients, we obtain:
\begin{equation}
    m_{w}=V \rho_{a l c} \rho_{w}\left(1-c_{\%}\right) /\left(\rho_{w} c_{\%}+\left(1-c_{\%}\right) \rho_{a l c}\right)
\end{equation}
\begin{equation}
    m_{a l c}=\frac{1-c_{\%}}{c_{\%}} m_{w}
\end{equation}
Thus, the number of alcohol molecules $N_{alc}$ and the number of water molecules $N_w$
can be obtained as follows:
\begin{equation}
    N_{a l c}=N_{A} \frac{m_{a l c}}{M_{a l c}}
\end{equation}
\begin{equation}
    N_{w}=N_{A} \frac{m_{w}}{M_{w}}
\end{equation}
Furthermore, $N_A$ is the Avogadro number, $M_{alc}$ is the molar mass of alcohol and 
$M_w$ is the molar mass of water.

The method is validated by taking the calculated number of molecules to 
calculate the mass concentration. The error between calculated mass 
concentrations and input concentrations do not surpass 2\%. Thus, the method 
is valid. The error arises from the fact that an integer number of molecules 
is required.\\
The densities of pure liquid are taken from online chemistry database CHERIC 
(Chemical Engineering and Materials Research Information Center) under 
conditions of 1 atm and 293K. Thus, the temperature of systems are set at 
293K and 1 atm pressure.\\
The thermal conductivity can be deconstructed into 2 terms: virial and 
convective. The virial terms represents the interatomic interaction 
contribution to the thermal conductivity while the convective terms show the 
diffusion in the thermal conductivity. It is shown that the virial terms 
dominate the convective terms and can be used to represent the trend of 
values \cite{lin_constructing_2011}. Furthermore, in LAMMPS, the calculation time of 
the virial terms is much shorter compared to the total thermal conductivity 
due to the absence of the enthalpy quantity that is required to calculate the 
total term. Since the main aim of this work is the trend of values instead of 
the absolute values themselves, to reduce calculation costs, only the virial 
term is calculated. However, we can still calculate the total terms for 
viscosity since we do not need enthalpy quantity for viscosity calculations.

